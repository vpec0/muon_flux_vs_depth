\documentclass{article}

\usepackage{graphicx}
\usepackage{fullpage}

\usepackage[backend=biber,
sorting=none,
%citestyle=numeric-comp,
style=numeric-comp]{biblatex}
\addbibresource{flux_depth_plot.bib}

\begin{document}

\begin{figure}[ht]
  \centering
%   \includegraphics[width=0.48\textwidth]{../plot/total_mu_flux_vs_vertical_depth}
  \includegraphics[width=12cm]{../plot/total_mu_flux_vs_vertical_depth_w_prediction}
  \caption{
Muon flux as a function of depth in metres water equivalent (w.e.) for
laboratories around the world. Black points represent measurements in
laboratories with flat overburden and blue points represent
laboratories under mountains. The red curve is based on MC simulations
of muons propagating through `standard rock'
\cite{Kudryavtsev:2008qh}. The red open diamond represents estimated
flux in the potential future space at Boulby mine at the level of
1,400~m (3,575~{m~w.e.}). The data points represents the
following measurements:
%
\textbf{CallioLab} (Pyh\"{a}salmi, Finland) at various depths
\cite{ENQVIST2005286},
\textbf{LSC} (Canfranc, Spain) \cite{Trzaska2019} (depth taken from
\cite{Canfranc2005}), \textbf{Soudan} (MN, USA)
\cite{PhysRevD.90.122003},
\textbf{Kamioka} (Japan) \cite{PhysRevD.93.012004} (conversion from
muon rate to flux based on MC simulations from
\cite{PhysRevD.74.053007}),
\textbf{Boulby} at 1,100~m level (2850~m~w.e.) \cite{REICHHART201367},
\textbf{LNGS} (Gran Sasso, Italy) \cite{Agostini.2019},
\textbf{SURF} (Sudbury, Canada) \cite{ABGRALL201770} (depth taken from
\cite{PhysRevD.27.1444}),
\textbf{LSM} (Modane, France) \cite{PhysRevD.40.2163},
\textbf{Sudbury} \cite{PhysRevD.80.012001},
\textbf{Jingping} \cite{Guo.2021}.
}
\label{fig:flux_depth_prediction}
\end{figure}



\textbf{Soudan} \cite{PhysRevD.90.122003}:\\
Total flux quoted as through horizontal surface. (Only muons with
energies above 1~GeV)

\textbf{Kamioka} \cite{PhysRevD.93.012004,PhysRevD.74.053007}:\\
Publication \cite{PhysRevD.93.012004} only quotes the rate of detected
muons. Publication \cite{PhysRevD.74.053007} quotes MC simulations
which translate muon flux to detection rate. This result was used to
convert the measured rate to the total muon flux as seen by a spherical
detector.

\textbf{Boulby} \cite{REICHHART201367}:\\
Measurement by ZEPLIN-III with total flux quoted as through a spherical detector.

\textbf{Gran Sasso} \cite{Agostini.2019}:\\
Measurement by Borexino, flux quoted as through a spherical detector.

\textbf{Fr\'ejus} \cite{PhysRevD.40.2163}:\\
Flux quoted as from `single muon events' and averaged over muon
directions. This can be, most likely, interpreted as meaning muon flux
as seen by spherical detector.

\textbf{Sudbury} \cite{PhysRevD.80.012001}:\\

\textbf{Jingping} \cite{Guo.2021}\\

\textbf{SURF} flux \cite{ABGRALL201770}, depth \cite{PhysRevD.27.1444}\\

\textbf{Canfranc} flux \cite{Trzaska2019}, depth \cite{Canfranc2005}\\

\textbf{Pyhasalmi} \cite{ENQVIST2005286}\\

Red curve \cite{Kudryavtsev:2008qh}.

% Troubles: can't tell what conventions different measurements used. I
% would assume the standard would be intensity in a horizontal flat
% detector.


\printbibliography

% \begin{thebibliography}{10}
% \bibitem{Borexino} G. Bellini, et al., Cosmic-muon flux and annual
%   modulation in Borexino at 3800 m water-equivalent depth, Journal of
%   Cosmology and Astroparticle Physics 05 (2012) 015.
% \end{thebibliography}

\end{document}